\documentclass{beamer}
\usetheme{CambridgeUS}
\usepackage{listings}
\usepackage{blkarray}
\usepackage{listings}
\usepackage{subcaption}
\usepackage{url}
\usepackage{tikz}
\usepackage{tkz-euclide} % loads  TikZ and tkz-base
%\usetkzobj{all}
\usetikzlibrary{calc,math}
\usepackage{float}
\renewcommand{\vec}[1]{\mathbf{#1}}
\usepackage[export]{adjustbox}
\usepackage[utf8]{inputenc}
\usepackage{amsmath}
\usepackage{amsfonts}
\usepackage{tikz}
\usepackage{hyperref}
\usepackage{bm}
\usetikzlibrary{automata, positioning}
\providecommand{\pr}[1]{\ensuremath{\Pr\left(#1\right)}}
\providecommand{\mbf}{\mathbf}
\providecommand{\qfunc}[1]{\ensuremath{Q\left(#1\right)}}
\providecommand{\sbrak}[1]{\ensuremath{{}\left[#1\right]}}
\providecommand{\lsbrak}[1]{\ensuremath{{}\left[#1\right.}}
\providecommand{\rsbrak}[1]{\ensuremath{{}\left.#1\right]}}
\providecommand{\brak}[1]{\ensuremath{\left(#1\right)}}
\providecommand{\lbrak}[1]{\ensuremath{\left(#1\right.}}
\providecommand{\rbrak}[1]{\ensuremath{\left.#1\right)}}
\providecommand{\cbrak}[1]{\ensuremath{\left\{#1\right\}}}
\providecommand{\lcbrak}[1]{\ensuremath{\left\{#1\right.}}
\providecommand{\rcbrak}[1]{\ensuremath{\left.#1\right\}}}
\providecommand{\abs}[1]{\vert#1\vert}

\newcounter{saveenumi}
\newcommand{\seti}{\setcounter{saveenumi}{\value{enumi}}}
\newcommand{\conti}{\setcounter{enumi}{\value{saveenumi}}}
\usepackage{amsmath}
\setbeamertemplate{caption}[numbered]{}                               
                               

\title{AI1110 Assignment 6}
\author{DEEPSHIKHA-CS21BTECH11016}
\date{\today}
\logo{\large \LaTeX{}}


\begin{document}
\begin{frame}
		\titlepage
	\end{frame}

\begin{frame}{Outline}
  \tableofcontents
\end{frame}
\section{Abstract}
\begin{frame}{Abstract}
\begin{itemize}
\item 	This document contains the solution to Question of Chapter 5 of Papoulis book.
\end{itemize}
\end{frame}
	
\section{Question}
\begin{frame}{Question}
\begin{block}{\textbf{ Exercise 5.19}}
If $X$ is an exponential random variable with parameter $\lambda$. Show that $Y$=$X^{\frac{1}{\beta}}$ has a Weibull distribution. 
\end{block}
\end{frame}
	
\section{Theory}
\begin{frame}{Theory}
\begin{block}{\textbf{Weibull Distribution}}
\begin{enumerate}
    \item Probability density function
    \begin{align}
        f(x)=\kappa \lambda^{\kappa} x^{\kappa-1} e^{{(-\lambda x)}^{\kappa}}
    \end{align}
  where, $\lambda$=Positive scale parameter 
  
     and  $\kappa$=Positive shape parameter
     \item Exponential Distribution is a special case when $\kappa$ = 1.
     \item Rayleigh Distribution is a special case when $\kappa$ = 2.
\end{enumerate}

\end{block}

\end{frame}
	
\section{Solution}
\begin{frame}{Solution}
Let the random variable X have the exponential distribution with probability density
function,
\begin{align}
   f_X(x)&=\lambda e^{\lambda x} u(x)\\
   Y=X^{\frac{1}{\beta}} &\implies x_1=y^{\beta}\\
   |\frac{dy}{dx}|&=\frac{1}{\beta} x^\frac{1}{\beta -1}\\
   \implies|\frac{dy}{dx}|&=\frac{1}{\beta} y^{1-\beta}\\
   f_Y(y)&=\frac{1}{|\frac{dy}{dx}|}f_X(x_1)\\
         &=\lambda \beta y^{\beta -1} e^{-\lambda y^{\beta}} U(y)
\end{align}
and it represents a Weibull distribution.
\end{frame}
\end{document}
